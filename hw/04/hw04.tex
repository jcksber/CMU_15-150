% hw04.tex 
% Created by Jack Kasbeer (jkasbeer) on September 28, 2015
%
% Principles of Functional Programming
% 15-150
% Assignment 04
\documentclass[11pt]{article}

\usepackage{amsmath}
\usepackage{amsthm}
\usepackage{amssymb}
\usepackage{fancyhdr}

\oddsidemargin0cm
\topmargin-2cm     
\textwidth16.5cm   
\textheight23.5cm  

\newcommand{\question}[2] {\vspace{.25in} \hrule\vspace{0.5em}
\noindent{\bf #1: #2} \vspace{0.5em}
\hrule \vspace{.10in}}
\renewcommand{\part}[1] {\vspace{.10in} {\bf (#1)}}

\newcommand{\myname}{Jack Kasbeer}
\newcommand{\myandrew}{jkasbeer@andrew.cmu.edu}
\newcommand{\myhwnum}{04}

\setlength{\parindent}{0pt}
\setlength{\parskip}{5pt plus 1pt}
 
\pagestyle{fancyplain}
\lhead{\fancyplain{}{\textbf{HW\myhwnum}}}     
\rhead{\fancyplain{}{\myname\\ \myandrew}}
\chead{\fancyplain{}{15-150}}

\begin{document}

\medskip            
\thispagestyle{plain}
\begin{center} 
{\Large 15-150 Assignment \myhwnum} \\
\myname \\
\myandrew \\
Section K \\
September 29, 2015 \\
\end{center}

% ##### TASK 2 ##### %
\question{2}{Parentheses Matching}
\begin{enumerate}
% Task 2.1 
\item Prove: for all values \verb|A : paren list, n : int,| such that \verb|n >= 0| and \verb|B : paren list|, if \verb|A| is balanced, then 
\begin{center} \verb|is_balanced(A @ B, n) = is_balanced(B, n)| \end{center}
\begin{proof}
By strong induction on $length$ of A.\\
Let $P(n)$ be the proposition that \verb|is_balanced(A @ B, n) = is_balanced(B, n)| for all \verb|n >= 0| and balanced integer lists \verb|A|, where \verb|n = length A|.\\
Base Case: \verb|length(A) = 0|.  It's trivially true that \verb|is_balanced(A @ B, 0) = is_balanced(B, 0)| since \verb|A@B = B|.\\
Inductive Hypothesis: Assume $P(k)$ is true for all integer lists with $length \neq 1$.\\
Considering $P(k+1)$...\\
\verb|is_balanced(A @ B, k) = is_balanced((L::A' @ R::A'') @ B, k)|, by def. of balanced paren list.\\
\verb|=> is_balanced(L::(A'@R::A''@B), k)|, by lemma 4\\
\verb|=> is_balanced(A'@(R::A''@B), k+1)|, by proven spec of \verb|is_balanced|\\
\verb|=> is_balanced(R::A''@B, k+1)|, by IH\\
\verb|=> is_balanced(A''@B, k)|, by \verb|is_balanced|\\
\verb|=> is_balanced(B, k)|, by IH\\
It follows that for all values \verb|A : paren list, n : int,| such that \verb|n >= 0| and \verb|B : paren list|, if \verb|A| is balanced, then 
\begin{center} \verb|is_balanced(A @ B, n) = is_balanced(B, n)| \end{center} 
\end{proof}
% Task 2.2 
\item Show: If \verb|L : paren list| is a balanced list, then \verb|is_balanced(L, 0) = true| 
\begin{proof}
Let \verb|L | be a balanced list.  Then, for some balanced lists \verb|M1| and \verb|M2|, \verb|L = (Left::M1) @ (Right::M2)|, by def. of a balanced paren list.\\
\verb|is_balanced(L, 0) = is_balanced((Left::M1) @ (Right::M2), 0)|, by Task 2.1\\
\verb|=> is_balanced(Left::M1, 0) = is_balanced(M1, 0+1)|, but by def. M1 is balanced so this is \verb|true|\\
and \verb|is_balanced(Right::M2, 1) = if n > 0 then is_balanced(M2, 0) else false|\\
\verb|=> is_balanced(Right::M2, 1) = is_balanced(M2, 0)| and M2 is also defined to be balanced.\\
Hence, if \verb|L : paren list| is a balanced list, then \verb|is_balanced(L, 0) = true|
\end{proof}
% Task 2.3 
\item \verb|pmatch| is implemented in \verb|hw04.sml|
\end{enumerate}

% ##### TASK 3 ##### %
\question{3}{Full Trees}
\begin{enumerate}
% Task 3.1 
\item \verb|shape : tree * int -> tree|
% Task 3.2 
\item Work to evaluate \verb|shape(T, n)|\\
$W_{shape}(Empty, n) = c_0$\\
$W_{shape}(T^m, n) = c_1 + W_{shape(T-left, n+1)} + W_{shape(T-right, n+1)}$\\
$W_{shape}(T^m, n) = c_1 + 2*W_{shape(T^{m-1}, n+1)}$\\
$W_{shape}(T^m, n) = c_2 + 4*W_{shape(T^{m-2}, n+2)}$\\
$W_{shape}(T^m, n) = O(2^m)$
% Task 3.3 
\item Span to evaluate \verb|shape(T,n)|\\
$S_{shape}(Empty, n) = c_0$\\
$S_{shape}(T^m, n) = c_1 + max(S_{shape(T^{m-1}, n+1)}, S_{shape(T^{m-1}, n+1)})$\\
$S_{shape}(T^m, n) = c_1 + S_{shape(T^{m-1}, n+1)}$\\
$S_{shape}(T^m, n) = O(m)$
% Task 3.4
\item \verb|census| is implemented in \verb|hw04.sml|
\end{enumerate}

% ##### TASK 4 ##### %
\question{4}{Quicksorting a List}
\begin{enumerate}
% Task 4.1
\item \verb|part| is implemented in \verb|hw04.sml|
% Task 4.2
\item Try to prove: For all integer lists, \verb|L|, \verb|quicksort L| evaluates to a \verb|<-|sorted permutation of \verb|L|.\\
The problem in the previous implementation was that it was using the first element as a basis for the sort.  The proof breaks down when trying to prove that \verb|x| is in the right position in the final \verb|<-|sorted list.
% Task 4.3
\item Fixed definition of \verb|quicksort| is in \verb|hw04.sml|.
\end{enumerate}

% ##### TASK 5 ##### %
\question{5}{Tree Traversal}
\begin{enumerate}
% Task 5.1
\item \verb|traversal| is implemented in \verb|hw04.sml|.
\end{enumerate}

\newpage

% ##### TASK 6 ##### %
\question{6}{Tree Size}
\begin{enumerate}
% Task 6.1
\item Prove: for all values \verb|t : tree|
\begin{center}\verb|size(t) = length(treeToList t)|\end{center}
\begin{proof}
By structural induction on \verb|T|.\\
Let $P(T)$ be the proposition that 
\begin{center}\verb|size(T) = length(treeToList(T))|\end{center}
Base Case: $P(Empty) \Rightarrow$ \verb|size(Empty) = 0 = length(treeToList(T)|, trivially.\\
Inductive Hypothesis: Let $T = (L, x, R)$ and assume $P(L)$ and $P(R)$ are true for some tree with a non-negative depth.\\
WWTS $P(T)$ is true for all integers $x \geq 0$.\\
\verb|Node(L, x, R)) = size(L) + 1 size(R)|, by def. of \verb|size|\\
\verb|= 1 + length(treeToList(L)) + length(treeToList(R))|, by IH\\
\verb|= length(treeToList(L) + x::treeToList(R))|, by Lemma 2\\
\verb|= length(treeToList(L) @ x::treeToList(R))|, by Lemma 2\\
\verb|= length(treeToList(Node(L,x,R)))|, by def. of \verb|treeToList|\\
Hence, by structural induction on \verb|T|,  for all values \verb|t : tree|, \verb|size(t) = length(treeToList t)|
\end{proof}
\end{enumerate}

% ##### TASK 7 ##### %
\question{7}{Heaps and Heaps of Heaps}
\begin{enumerate}
% Task 7.1
\item \verb|insert| is implemented in \verb|hw04.sml|.
% Task 7.2
\item \verb|join| is implemented in \verb|hw04.sml|.
% Task 7.3
\item \verb|heapify| is implemented in \verb|hw04.sml|.
\end{enumerate}

\end{document}


















